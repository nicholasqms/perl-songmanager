              
\documentclass[12pt]{article}
\usepackage[brazilian]{babel}
\usepackage[utf8]{inputenc}
\usepackage{fullpage}
\usepackage{multicol}
\usepackage{listings}
\usepackage{color}

\title{Busca de expressões em músicas usando Perl}
\author{Alunos: Krishynan Shanty, Nicholas Quagliani}

\definecolor{mygreen}{rgb}{0,0.6,0}
\definecolor{mygray}{rgb}{0.5,0.5,0.5}
\definecolor{mymauve}{rgb}{0.58,0,0.82}

\lstset{ %
  language=Perl,
  basicstyle=\footnotesize,        % size of fonts used for the code
  breaklines=true,                 % automatic line breaking only at whitespace
  captionpos=b,                    % sets the caption-position to bottom
  frame=lrtb,
  columns=fullflexible,
  escapeinside={[*}{*]},
  commentstyle=\color{mygreen},    % comment style
  keywordstyle=\color{blue},       % keyword style
  stringstyle=\color{mymauve},     % string literal style
}

\begin{document}

\maketitle

\section{Introdução}
A linguagem Perl oferece muitos recursos para manipulação de arquivos e expressões regulares. Utilizamos disso para construção de um módulo com funções para extrair informações de interesse de um modelo de arquivo de música e para formulação de uma função de busca entre todas músicas em um diretório.
\section{Definição do Modelo}
O modelo de uma música deve ser, onde a terceira linha é a data de lançamento da música:
\begin{lstlisting}
Autor:Exemplo de Autor
Titulo:Exemplo de Titulo
01/01/0000
Letra:Exemplo de Letra
Permite novas linhas

E tambem ate UMA linha vazia
\end{lstlisting}
A primeira função foi feita para extrair o autor do arquivo de texto. A função recebe como parâmetro um escalar contendo todas as linhas do arquivo. Todas as funções utilizam a comparação sem se importar com maiúsculas ou minúsculas.
\begin{lstlisting}
sub EncontraAutor{
	my $arquivo = $_[0];
	my ($autor);
	
	($autor) = ($arquivo =[*$\sim$*] /^Autor:([^\n\r]*)/i); #No inicio do arquivo,captura tudo ate o pula linha
	return $autor;
}
\end{lstlisting}
Analogamente,a segunda função foi feita para extrair o titulo do arquivo de texto. A função recebe como parâmetro um escalar contendo todas as linhas do arquivo. \\
\begin{lstlisting}
sub EncontraTitulo{
	my $arquivo = $_[0];
	my ($titulo);
	
	($titulo) = ($arquivo =[*$\sim$*] /[\n\r]Titulo:([^\n\r]*)/i); #Dessa vez esta apos um pula linha, captura tudo ate o proximo pula linha
	return $titulo;
}
\end{lstlisting}
A função da data basta pegar o padrão de dígitos e barras pois não ocorrerá em outro local do arquivo. Ela devolve o dia,mês e ano como 3 escalares separados.
\begin{lstlisting}
sub encontraData{
	my $arquivo = $_[0];
	my ($dia,$mes,$ano);
	
	($dia, $mes, $ano) = ($arquivo =[*$\sim$*] /(\d\d)\/(\d\d)\/(\d\d\d\d)/);
	#Dois digitos seguidos de barra seguido de dois digitos seguido de barra seguido de quatro digitos
	return ($dia,$mes,$ano);
}
\end{lstlisting}
A função de encontrar letra ou trecho recebe então dois argumentos, o escalar com as linhas do arquivo e um segundo argumento com a palavra ou trecho a ser procurado. A função retira os caracteres pula linha para garantir que o trecho será encontrado mesmo que na organização da letra, estejam em linhas diferentes.

\begin{lstlisting}
sub EncontraLetra{
	my $arquivo = $_[0];
	my $trecho = $_[1];
	my $arquivoUmaLinha = $arquivo;
	my $trechoCompleto;
	
	$arquivoUmaLinha =[*$\sim$*] tr{\n\n}{ };
	$arquivoUmaLinha =[*$\sim$*] tr{\n}{ };
	if ($arquivoUmaLinha =[*$\sim$*] /\sLetra:(.*$trecho.*)/i) {
		($trechoCompleto) = $1 =[*$\sim$*] /((?:['\w]+(?:,?\s)){0,10}$trecho(?:(?:,?\s)['\w]+){0,10})/i; #regex bolado pra copiar um trecho com ate 10 palavras ao redor da palavra achada podendo ter virgula
		return $trechoCompleto;
	}
	return 0;
}
\end{lstlisting}
A função de encontrar techo duplo recebe um argumento a mais sendo o segundo trecho. Alem disso, há uma lógica para definir o trecho retornado. Se a quantidade de caracteres entre os dois trechos digitados for menor que 40 ela retorna o trecho iniciando com a primeira palavra e terminando com a segunda. Caso contrário, faz como a EncontraLetra e envia os trechos separadamente mas menores entre cada trecho.\\\\
\begin{lstlisting} 
sub EncontraLetraDupla{
	my $arquivo = $_[0];
	my $trecho1 = $_[1];
	my $trecho2 = $_[2];
	my $arquivoUmaLinha = $arquivo;
	my $trechoCompleto;
	my $aux;
	
	$arquivoUmaLinha =[*$\sim$*] tr{\n\n}{ };
	$arquivoUmaLinha =[*$\sim$*] tr{\n}{ };
	if ($arquivoUmaLinha =[*$\sim$*] /\sLetra:.*($trecho1(.*)$trecho2).*/i) {
		if (length ($2) < 40) {      #envia o trecho entre as palavras procuradas se nao for grande
			return $1;
		}
		($trechoCompleto) = $arquivoUmaLinha =[*$\sim$*]
		 /((?:['\w]+(?:,?\s)){0,6}$trecho1(?:(?:,?\s)['\w]+){0,6})/i; #se nao procura ate mais 6 palavras entre o trecho
		$trechoCompleto .= " (...) ";
		($aux) = $arquivoUmaLinha =[*$\sim$*] /((?:['\w]+(?:,?\s)){0,6}$trecho2(?:(?:,?\s)['\w]+){0,6})/i;
		$trechoCompleto .= $aux;
		return $trechoCompleto;
	}
	return 0;
}
\end{lstlisting}
A função final é a de pesquisa global. Para escolher o tipo de pesquisa, o primeiro argumento é um escalar representando uma string. A função aceita que as palavras chave (autor,titulo,lancamento,data,pedaco e dupla) sejam escritas por extenso ou apenas a primeira letra. Ela percorre todos os arquivos de texto na pasta songs e faz a pesquisa nos válidos. No caso de autor, se for encontrado o nome exato do artista, todas músicas daquele artista são impressas. Caso o trecho seja uma expressão não exata de outros artistas, o nome deles será sugerido garantindo que não há repetição.
No título, se for exato imprime na tela todas informações (o arquivo inteiro) senão imprime os títulos possíveis. Para data está apenas implementado imprimir o título das músicas que tem o ano igual ao sugerido. Por último, na pesquisa por trecho ou trechos imprime o título das músicas e o trecho encontrado para ajudar no reconhecimento.
\begin{lstlisting} 
sub PesquisaGlobal{
my $tipoPesquisa = $_[0];
my $procurado = $_[1];
my $termo1 = $_[2];
my ($achados,$atual,$dia,$mes,$ano);
$achados = "";

	foreach my $arquivos (glob("songs/*.txt")) {
		open FILE, $arquivos or die "Couldn't open file: $!";
		my $arquivo = do {local $/; <FILE>};
		
		if (!ValidaArquivo($arquivo)){
			if ($tipoPesquisa =[*$\sim$*] /a(?:utor)?/i) {
				$atual = EncontraAutor($arquivo);        
				if ($atual=[*$\sim$*] /^$procurado$/i) {   #se for exato,devolve todas musicas que forem com o autor exato
					$achados .= "Musica dele:";
					$achados .= EncontraTitulo($arquivo);
					$achados .= "\n";
				}
				else {
					if ($atual =[*$\sim$*] /[\w\s]*$procurado[\w\s]*/i) {  #se nao for exato mas ainda acha
						if (!($achados =[*$\sim$*] /[\w\s]*$atual[\w\s]*/i)){  #confere se ja nao sugeriu esse artista antes
							$achados .= "Artista possivel:$atual\n";
						}
					}
				}
			}
			if ($tipoPesquisa =[*$\sim$*] /t(?:itulo)?/i) {
				$atual = EncontraTitulo($arquivo);
				if ($atual=[*$\sim$*] /^$procurado$/i) {  #exato mostra a musica inteira
					$achados .= $arquivo;
					$achados .= "\n";
				}
				else {
					if ($atual =[*$\sim$*] /[\w\s]*$procurado[\w\s]*/i) {  #nao exato retorna o titulo
						$achados .= "Musica possivel:$atual\n";
					}
				}			
			}
			if ($tipoPesquisa =[*$\sim$*] /p(?:edaco)?/i) {
				$atual = EncontraLetra($arquivo,$procurado);
				if ($atual) {
					$achados .= "Musica:";
					$achados .= EncontraTitulo($arquivo);
					$achados .= "\nTrecho:\n$atual\n";
				}
			}
			if ($tipoPesquisa =[*$\sim$*] /d(?:upla)?/i) {
				$atual = EncontraLetraDupla($arquivo,$procurado,$termo1);
				if ($atual) {
					$achados .= "Musica:";
					$achados .= EncontraTitulo($arquivo);
					$achados .= "\nTrecho:\n$atual\n";
				}			
			}
			if ($tipoPesquisa =[*$\sim$*] /L(?:ancamento)?/i) {
				($dia,$mes,$ano) = EncontraData($arquivo);#nao uso o dia e o mes pra nada ainda, fica pro futuro
				if ($procurado==$ano) {
					$achados .= EncontraTitulo($arquivo);
					$achados .= "\n";
				}			
			}
		}
	}
	print $achados;
}
\end{lstlisting}
\section{Casos de uso}
As seguintes músicas foram adicionadas na pasta songs para testar as diferentes funcionalidades, as letras não estão completas para caber aqui. Os nomes de arquivo não importam:
\begin{multicols}{2}
\begin{lstlisting}[language={}]
Autor:Bee Gees
Titulo:Stayin' Alive
13/12/1977
Letra:Well, you can tell by the way I use my walk
I'm a woman's man, no time to talk.
Music loud and women warm,
I've been kicked around since I was born.
And now it's all right, it's okay,
you may look the other way.
We can try to understand
the New York Times' effect on man.

Whether you're a brother or whether you're a mother
you're stayin' alive, stayin' alive.
Feel the city breakin' and everybody shakin'
and you're stayin' alive, stayin' alive.
\end{lstlisting}
\columnbreak
\begin{lstlisting}[language={}]
Autor:Bee Gees
Titulo:How Deep is Your Love
01/10/1977
Letra:I know your eyes in the morning sun
I feel you touch me in the pouring rain
And the moment which you wander far from me
I wanna feel you in my arms again

And you come to me on a summer breeze
Keep me warm in your love then you softly leave
And it's me you need to show
How deep is your love
\end{lstlisting}
\end{multicols}
\begin{multicols}{2}
\begin{lstlisting}[language={}]
Autor:Michael Jackson
Titulo:Beat it
30/11/1982
Letra:They told him don't you ever come around here
Don't wanna see your face, you better disappear
The fire's in their eyes and their words are really clear
So beat it, just beat it.

You better run, you better do what you can
Don't wanna see no blood, don't be a macho man
You wanna be tough, better do what you can
\end{lstlisting}
\columnbreak
\begin{lstlisting}[language={}]
Autor:Michael Buble
Titulo:Everything
23/05/2007
Letra:You're a falling star
You're the getaway car
You're the line in the sand
When I go too far
You're the swimming pool
On an august day
And you're the perfect thing to say
And you play it coy but it's kinda cute
Oh when you smile at me
\end{lstlisting}
\end{multicols}
Temos duas músicas de artistas com Michael no nome e duas músicas do artista Bee Gees. Foi também adicionado um arquivo que não era válido. O seguinte programa simples foi feito para gerar os argumentos para a função PesquisaGlobal:
\begin{lstlisting}
use warnings;
use strict;
use songmanager ':all'; #adiciona o modulo criado com as funcoes

my ($tipoPesquisa,$procurado,$termo1);
print "Bem vindo ao Song Manager.\n";
print "A(utor)-> Pesquisa por um autor. Retorna todas musicas do autor se for exato\n";
print "T(itulo)-> Pesquisa por um titulo. Retorna a musica inteira se for exato\n";
print "L(ancamento)-> Pesquisa por um ano de lancamento. Retorna os nomes das musicas\n";
print "P(edaco)-> Pesquisa por um trecho. Retorna o titulo e um pedac
o do trecho achado\n";
print "D(upla)-> Pesquisa por DOIS trechos desconectados. Retorna o titulo e um pedaco do trecho achado\n";
print "Digite por extenso ou a primeira letra da opcao desejada:\n";

$tipoPesquisa = <>;

if (!$tipoPesquisa =~ /^[atlpd]/i) {
	print "Opcao invalida";
}
if ($tipoPesquisa =~ /d/i) {
	print "Digite o primeiro trecho pra buscar:";
	chomp($procurado = <STDIN>);
	print "Digite o segundo trecho pra buscar:";
	chomp($termo1 = <STDIN>);
	PesquisaGlobal(substr($tipoPesquisa, 0, 1),$procurado,$termo1);
     #so mandei a primeira letra que ai a pessoa pode digitar errado. so pra testar mesmo
}
else {
	print "Digite o que ira buscar:";
	$procurado = <>;
	PesquisaGlobal(substr($tipoPesquisa	, 0, 1),$procurado);
}
\end{lstlisting}
Exemplos de pesquisa, o símbolo \textgreater\textgreater\space representa que o usuário digitou o que está a seguir:
\begin{lstlisting}[language={}]
C:\Users\Krishynan\Desktop\perl-songmanager>perl testaModulo.pl
Bem vindo ao Song Manager.
A(utor)-> Pesquisa por um autor. Retorna todas musicas do autor se for exato
T(itulo)-> Pesquisa por um titulo. Retorna a musica inteira se for exato
L(ancamento)-> Pesquisa por um ano de lancamento. Retorna os nomes das musicas
P(edaco)-> Pesquisa por um trecho. Retorna o titulo e um pedaco do trecho achado
D(upla)-> Pesquisa por DOIS trechos desconectados. Retorna o titulo e um pedaco do trecho achado
Digite por extenso ou a primeira letra da opcao desejada:
>>A
Digite o que ira buscar:>>MICHAEL
Artista possivel:Michael Jackson
Artista possivel:Michael Buble
\end{lstlisting}
\begin{lstlisting}[language={}]
C:\Users\Krishynan\Desktop\perl-songmanager>perl testaModulo.pl
Bem vindo ao Song Manager.
A(utor)-> Pesquisa por um autor. Retorna todas musicas do autor se for exato
T(itulo)-> Pesquisa por um titulo. Retorna a musica inteira se for exato
L(ancamento)-> Pesquisa por um ano de lancamento. Retorna os nomes das musicas
P(edaco)-> Pesquisa por um trecho. Retorna o titulo e um pedaco do trecho achado
D(upla)-> Pesquisa por DOIS trechos desconectados. Retorna o titulo e um pedaco do trecho achado
Digite por extenso ou a primeira letra da opcao desejada:
>>a
Digite o que ira buscar:>>bee gees
Musica dele:Stayin' Alive
Musica dele:How Deep is Your Love
\end{lstlisting}
\begin{lstlisting}[language={}]
C:\Users\Krishynan\Desktop\perl-songmanager>perl testaModulo.pl
Bem vindo ao Song Manager.
A(utor)-> Pesquisa por um autor. Retorna todas musicas do autor se for exato
T(itulo)-> Pesquisa por um titulo. Retorna a musica inteira se for exato
L(ancamento)-> Pesquisa por um ano de lancamento. Retorna os nomes das musicas
P(edaco)-> Pesquisa por um trecho. Retorna o titulo e um pedaco do trecho achado
D(upla)-> Pesquisa por DOIS trechos desconectados. Retorna o titulo e um pedaco do trecho achado
Digite por extenso ou a primeira letra da opcao desejada:
>>T
Digite o que ira buscar:>>stayin' Alive
Autor:Bee Gees
Titulo:Stayin' Alive
13/12/1977
Letra:Well, you can tell by the way I use my walk
I'm a woman's man, no time to talk.
Music loud and women warm,
I've been kicked around since I was born.
And now it's all right, it's okay,
you may look the other way.
We can try to understand
\end{lstlisting}
\begin{lstlisting}[language={}]
C:\Users\Krishynan\Desktop\perl-songmanager>perl testaModulo.pl
Bem vindo ao Song Manager.
A(utor)-> Pesquisa por um autor. Retorna todas musicas do autor se for exato
T(itulo)-> Pesquisa por um titulo. Retorna a musica inteira se for exato
L(ancamento)-> Pesquisa por um ano de lancamento. Retorna os nomes das musicas
P(edaco)-> Pesquisa por um trecho. Retorna o titulo e um pedaco do trecho achado
D(upla)-> Pesquisa por DOIS trechos desconectados. Retorna o titulo e um pedaco do trecho achado
Digite por extenso ou a primeira letra da opcao desejada:
>>p
Digite o que ira buscar:>>love
Musica:How Deep is Your Love
Trecho:
me on a summer breeze Keep me warm in your love then you softly leave And it's me you need to
Musica:Everything
Trecho:
We'll see it through And you know that's what our love can do
\end{lstlisting}
\section{Conclusão}
As funções foram implementadas com sucesso, permitindo a análise de vários arquivos no diretório. Foi decidido a metodologia da comparação exata ou não para evitar que imprimisse todas as músicas possiveis e as letras iriam poluir toda a informação. Da forma implementada, o usuário confirma como é o nome exato da música e basta digitá-lo inteiro para acessar a letra.
\end{document}
              